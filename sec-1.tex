\part{源代码管理}
\begin{frame}[<+->]{引子}
\begin{itemize}
  \item 没有源代码管理的世界
  \item 修订控制/版本管理
  \item 谁需要源代码管理
  \item 使用源代码管理的代价
  \item 使用源代码管理的好处
  \item 有哪些源代码管理工具
\end{itemize}
\end{frame}

\part{独自使用Git的情景}
\begin{frame}[<+->][fragile]{安装Git}
\begin{itemize}
  \item Linux下的安装
  不必废话
\begin{Verbatim}[frame=single,commandchars=\\\{\}]
sudo apt-get install git-core
\end{Verbatim}

  \item Windows下
  使用msisgit
\end{itemize}
\end{frame}

\begin{frame}[<+->][fragile]{配置}
\begin{Verbatim}[frame=single,commandchars=\\\{\}]
git config --global user.name "Your Name"
git config --global user.email "you@example.com"
git config --global core.editor vim
git config --global core.quotepash false
\end{Verbatim}
\end{frame}

\begin{frame}[<+->][fragile]{用起来}
\onslide<+->
新建一个项目
\begin{Verbatim}[frame=single,commandchars=\\\{\}]
mkdir xproj
cd xproj
vi main.c
\end{Verbatim}

\onslide<+->
为这个项目建立一个Git仓库并提交第一个版本
\begin{Verbatim}[frame=single,commandchars=\\\{\}]
git init
git add .
git commit -m 'The first of first commit.'
\end{Verbatim}
\end{frame}


