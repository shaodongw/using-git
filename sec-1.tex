\part{UNIX 简介}
\section{学习指导}
\begin{frame}{引子}
\begin{itemize}
  \item UNIX 是什么?
  \item 这门课讲什么?
  \item 为什么学习UNIX?
  \item 能够从这门课中得到什么?
  \item 如何学习UNIX?
  \item 这门课如何安排?
\end{itemize}
\end{frame}

\begin{frame}{UNIX}
\begin{itemize}
  \item UNIX 是一类操作系统
  \item 应用于所有类型的计算机
  \item 包含众多的变体
  \item 诞生于1969年,贝尔实验室
\end{itemize}
\end{frame}

\begin{frame}{内容}
\begin{itemize}
  \item UNIX的用户环境
  \begin{itemize}
    \item 普通用户
    \item 程序员
  \end{itemize}
  \item UNIX基础
    \begin{itemize}
      \item 命令
      \item 实用工具
    \end{itemize}
  \item 文件系统
  \item shell
\end{itemize}
\end{frame}

\begin{frame}{动机}
\begin{itemize}
  \item UNIX是最经典的操作系统
  \item 广泛应用于各种领域
  \item 富有特色的命令和实用工具集合
  \item UNIX是程序员的操作系统
  \item 拥有独特的技术魅力
  \item 深远的技术文化影响
\end{itemize}
\end{frame}

\begin{frame}{目标}
\begin{itemize}
  \item 学习UNIX环境中的基本概念
  \item 使用UNIX的常用命令和使用工具
  \item 学习shell(重定向,管道,文件名替换)
  \item 了解正则表达式
  \item 熟悉一种文本编辑工具
\end{itemize}
\end{frame}

\begin{frame}{方法}
\begin{itemize}
  \item 注意总结规律
  \item 重视实验和练习
  \item 增加课外训练
  \item 勇于尝试
\end{itemize}
\end{frame}

\begin{frame}[<+->]{建立自己的练习环境}
    \begin{itemize}
        \item 获得 Live CD 的镜像,并写到 CDROM 光盘上。
        \item 制作一只可以启动的 USB 盘。(或者移动硬盘)
        \item 在 Windows 操作系统上,安装虚拟机软件(vmware, VirtualPC, VirtualBox),在虚拟机上安装 Linux 系统。
        \item 在 Windows 分区上,安装一套独立的 Linux 系统。(整个 Linux 系统,作为一个 Windows 的文件。wubi)
        \item 调整硬盘的分区,在空闲的分区上,安装 Linux 操作系统。(双系统)
        \item 在整个硬盘上,安装 Linux 系统。(为了偶尔使用Windows,可以在 Linux 系统上安装虚拟机,在虚拟机当中,安装 Windows 系统。)
    \end{itemize}
\end{frame}

\begin{frame}{发行版本的选择}
Linux 就是一种 UNIX.

可以在PC机上运行,是最好的学习环境。

Ubuntu, Fedora, Debian, SuSe …… 等等,分别是不同的~Linux 发行套件。

你还可以选择 FreeBSD 作为学习 UNIX 的环境。

参考:distrowatch.com
\end{frame}

\begin{frame}{计划}
\begin{itemize}
  \item 讲授:24学时
  \item 实验:8学时
  \item 课外练习:不少于32学时
\end{itemize}
\end{frame}

% \section{NIX 之名}
% \begin{frame}{UNIX 是什么?}
% \begin{itemize}
%   \item Microsoft Windows 是一种操作系统
%   \item Mac OS X 是一种操作系统
%   \item UNIX 是一种操作系统
%   \item $\mu{}$COSII 是一种操作系统
%   \item PalmOS 是一种操作系统
%   \item Symbian 是一种操作系统
% \end{itemize}
% \end{frame}

\section{操作系统}
\begin{frame}{操作系统}
操作系统是负责管理计算机系统软硬件资源,使用户简单、高效、公平、有序、安全地
使用计算机系统的系统软件。
\begin{itemize}
	\item 为用户和应用程序提供接口
	\item 管理并分配资源
	\item 加载并执行用户程序
\end{itemize}
\end{frame}

\begin{frame}
\begin{itemize} 
  \item 两类主要的用户:一般用户,程序员
  \item 两个主要的接口:AUI,API
  \onslide<2->{
  \begin{itemize}
    \item Application User's Interface
    \item Application Programmer's Interface
  \end{itemize}}
  \item 四个层次
  \onslide<3->{
  \begin{itemize}
    \item UNIX shell, 命令和应用程序
    \item 语言库和系统调用接口
    \item 操作系统内核
    \item 计算机硬件
  \end{itemize}}
\end{itemize}
\end{frame}

\begin{frame}
\begin{tikzpicture}[
	layer/.style={
	rectangle,
	minimum height=12mm,
	minimum width=80mm,
	very thick,
	draw=red!50!black!50,
	top color=white,
	bottom color=red!50!black!20,
	},
	user/.style={
	circle,
	minimum size=12mm,
	very thick,
	draw,
	},
	help lines/.style={color=blue!50,very thin},
	]
	%\draw[help lines,step=.5cm] (-5,-2) grid (7,7);

	\node [layer](hardware){计算机硬件};
	\node [layer,above=-1pt of hardware](kernel){操作系统内核};
	\node [layer,above=-1pt of kernel](server){语言库和系统调用接口};
	\node [layer,above=-1pt of server](shell){UNIX shell, 命令和应用程序};
    \node [right=0.2cm of shell,yshift=0.35cm]{AUI,用户接口};
    \node [right=0.2cm of server,yshift=0.35cm]{API,程序接口};
	
	\node [user,above=of shell,xshift=-35mm](user1){用户};
	\node [user,right=of user1](user2){用户};
	\node [right=of user2](dots){\ldots};
	\node [user,above=of shell,xshift=35mm](user3){用户};

	\draw [->] (user1) -- (user1 |- shell.north);
	\draw [->] (user2) -- (user2 |- shell.north);
	\draw [->] (user3) -- (user3 |- shell.north);
\end{tikzpicture}
\end{frame}

\begin{frame}
两种类型的用户界面:
\begin{itemize} 
  \item 字符用户界面,CUI
  \item 图形用户界面,GUI
\end{itemize}

\bigskip \pause
CUI, Character User Interface

GUI, Graphical User Interface
\end{frame}

\begin{frame}{UNIX 的特点}
    \begin{itemize}
    \item 多用户
    \item 多任务
    \item 层次型文件系统
    \item I/O 操作与设备无关
    \item 独特的用户界面:shell
    \item 丰富而完善的系统工具
    \item 出色的服务功能
    \item 支持多种硬件平台(UNIX是用C语言编写的)
    \end{itemize}
\end{frame} 

\begin{frame}{UNIX 的基本结构}
\begin{center}
\includegraphics[width=9cm,keepaspectratio]{figure/os_struc.jpg}
\end{center}
\end{frame}

\section{UNIX 的现状与历史}
\begin{frame}{UNIX 家族}
UNIX 是一类操作系统。
有时,我们会把这类操作系统称为“类 UNIX 操作系统”。

\begin{itemize}
  \item AIX (IBM)
  \item HP-UX (HP)
  \item Solaris (SUN, Oracle)
  \item FreeBSD, NetBSD, OpenBSD (Berkeley Software Distribution)
  \item GNU/Linux 
\end{itemize}
\end{frame}

\begin{frame}{UNIX 的简要历史}
\begin{itemize}
  \item 1969年~~诞生
  \item 1970年~~移植到PDP-11/20等系列的计算机上
  \item 1973年~~用高级语言C重写
  \item 1975年~~向教育机构提供
  \item 1978年~~功能趋于完善,出现两个重要分支
  \item 1983~~system V,4.2 BSD
  \item 1988~~标准化 POSIX.1
  \item 199x~~Linux
\end{itemize}
\end{frame}

% some important persons in the UNIX history.


%   \begin{frame}{讨论}
%   \begin{itemize}
%     \item UNIX 的主要版本有哪些?
%     \item 什么是内核?
%     \item 什么是 shell ?
%     \item 你的UNIX系统的版本号?
%   \end{itemize}
%   \end{frame}

\begin{frame}{练习}
在你自己的计算机上安装一套~Linux 
\end{frame}
