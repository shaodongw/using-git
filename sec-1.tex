\part{源代码管理简介}

\begin{frame}[<+->]{源代码管理工具}
  \begin{itemize}
    \item 源代码管理工具是什么?
    \item 我们为什么需要它?
    \item 不使用源代码管理会怎样?
    \item 常见的源代码管理工具:CVS,Subversion(SVN),Git
  \end{itemize}
\end{frame}

%   \begin{frame}[<+->]{引子}
%   \begin{itemize}
%     \item 没有源代码管理的世界
%     \item 修订控制/版本管理
%     \item 谁需要源代码管理
%     \item 使用源代码管理的代价
%     \item 使用源代码管理的好处
%     \item 有哪些源代码管理工具
%   \end{itemize}
%   \end{frame}

\begin{frame}[<+->]{以Git为例}
  \begin{itemize}
    \item Git 是一种完全分布式的源代码管理工具
    \item Git 的使用非常适合多个线索的并行开发
    \item Git 仓库存在于每个工作目录当中,天然的备份
    \item Git 是自由软件,拥有广泛用户基础 
  \end{itemize}
\end{frame}

\begin{frame}[<+->]{使用Git所要涉及的几个基本词汇}
  \begin{itemize}
    \item 仓库(repository)
    \item 工作目录(working directory)
    \item 提交(commit)
    \item 检出(checkout)
    \item 分支(branch)
    \item 合并(merge)
    \item 日志(log)
  \end{itemize}
\end{frame}

\begin{frame}[<+->]{版本管理的工作情境}
  \begin{itemize}
    \item 创建仓库
    \item 提交工作
    \item 推到远程
    \item 拖到本地
    \item 合并
    \item 提交
    \item 切换分支
  \end{itemize}
\end{frame}

\part{Git入门}

\begin{frame}[<+->][fragile]{检查Git是否已安装}
  \begin{itemize}
    \item Linux 系统中,使用命令:
\begin{Verbatim}[frame=single,commandchars=\\\{\}]
git --version
\end{Verbatim}
    \item Windows 系统下,可以看到如下图标:
        \begin{center}
            \rule{0mm}{25mm}
            \includegraphics<.->[height=20mm]{figure/git-logo.png}
            \hspace{10mm}
            \includegraphics<.->[height=20mm]{figure/msysgit-logo.png}
        \end{center}
    \item Mac 系统下(?)
  \end{itemize}
\end{frame}

\begin{frame}[<+->][fragile]{安装Git}
\begin{itemize}
  \item Linux下的安装
  不必废话
\begin{Verbatim}[frame=single,commandchars=\\\{\}]
sudo apt-get install git-core
\end{Verbatim}

  \item Windows 下载并安装 Git for Windows 或者 msysGit

  %(中文文件名还存在问题)
  
  \item Mac下的可选用Git for OS X,或者 GitX
\end{itemize}
\end{frame}

\begin{frame}[<+->][fragile]{配置}
\begin{Verbatim}[frame=single,commandchars=\\\{\}]
git config --global user.name "Your Name"
git config --global user.email "you@example.com"
git config --global core.editor vim
git config --global core.quotepath false
\end{Verbatim}
\end{frame}

\begin{frame}[<+->][fragile]{用起来}
\onslide<+->
新建一个项目
\begin{Verbatim}[frame=single,commandchars=\\\{\}]
mkdir xprj
cd xprj
vi main.c
\end{Verbatim}

\onslide<+->
为这个项目建立一个Git仓库并提交第一个版本
\begin{Verbatim}[frame=single,commandchars=\\\{\}]
git init
git add .
git commit -m 'The first of first commit.'
\end{Verbatim}
\end{frame}

\begin{frame}[<+->][fragile]{概念补充}
  \begin{itemize}
    \item 仓库在哪里?

    本地仓库的概念
    \item “加入”的含义

    Staging Area, Index, Cache 

    add, update, stage
    \item 查看Git状态
\begin{Verbatim}[frame=single,commandchars=\\\{\}]
git status
\end{Verbatim}

  \end{itemize}
\end{frame}


\begin{frame}[<+->][fragile]{修改与提交}
\onslide<+->
修改一些内容
\begin{Verbatim}[frame=single,commandchars=\\\{\}]
vi main.c
\end{Verbatim}

\onslide<+->
查看状态,提交修改
\begin{Verbatim}[frame=single,commandchars=\\\{\}]
git status
git add main.c
git status
git commit
\end{Verbatim}

\onslide<+->
合并操作步骤
\begin{Verbatim}[frame=single,commandchars=\\\{\}]
git commit -a -m 'Append a newline at the end of string'
\end{Verbatim}
\end{frame}

\begin{frame}[<+->][fragile]{增加文件}
\onslide<+->
增加两个文件
\begin{Verbatim}[frame=single,commandchars=\\\{\}]
vi common.c
vi common.h
\end{Verbatim}

\onslide<+->
查看状态,注意Untracked files提示
\begin{Verbatim}[frame=single,commandchars=\\\{\}]
git status
\end{Verbatim}

\onslide<+->
加到Git仓库中
\begin{Verbatim}[frame=single,commandchars=\\\{\}]
git add common.c common.h
git status
git commit -m 'Two files common.[ch] are added.'
git status
\end{Verbatim}
\end{frame}

\begin{frame}[<+->]{纳入/未纳入代码管理的文件}
  \begin{itemize}
    \item git add <file> 纳入Git的控制之下
    \item git rm <file> 从Git管理中删除文件
    \item 特殊文件\  .gitignore 专门设定无需Git管理的文件
  \end{itemize}
\end{frame}

\begin{frame}[<+->][fragile]{选择性提交}
\onslide<+->
继续修改文件
\begin{Verbatim}[frame=single,commandchars=\\\{\}]
vi main.c
vi common.c
\end{Verbatim}

\onslide<+->
忘记了修改过哪些文件
\begin{Verbatim}[frame=single,commandchars=\\\{\}]
git status
git diff
\end{Verbatim}

\onslide<+->
只想提交针对个别文件的修改
\begin{Verbatim}[frame=single,commandchars=\\\{\}]
git add main.c
git commit
\end{Verbatim}

\onslide<+->
所有修改过的文件,一律提交
\begin{Verbatim}[frame=single,commandchars=\\\{\}]
git commit -am 'fix the #755 bug'
\end{Verbatim}
\end{frame}

\begin{frame}[<+->]{Git仓库中包含每一次提交所形成的历史}
\begin{tikzpicture}
[ commit/.style={ circle, minimum size=11mm, draw=black, },
start chain,node distance=8mm, every node/.style={on chain,join}, every join/.style={<-}
]
\foreach \x in {C1,C2,C3,C4,C5}
{
    \visible<+->
    {\node [commit] {\x};
        \visible<.>{
        \begin{scope}[start branch=tip]
        \node (tip) [on chain=going above, node distance=3mm] {master};
        \end{scope}}}
}
\end{tikzpicture}
\end{frame}

\begin{frame}[<+->][fragile]{观察提交历史}
  \begin{itemize}
    \item 观察Git提交历史
    \begin{Verbatim}[frame=single,commandchars=\\\{\}]
git log    
    \end{Verbatim}
    \item 提交的名字
    \begin{itemize}
        \item 绝对提交名:533e3140bffee43b02c5648c8fcc3e63232739a6
        \item 绝对提交名的简写:533e31
        \item 参照名:master, dev, fixbug, v1.0
        \item 符号参照名:\verb|HEAD|
        \item 相对提交名:\verb|master^|
    \end{itemize}
  \end{itemize}
\end{frame}

\begin{frame}[fragile]{小结}
\onslide<+->
使用Git,线性进展,使用过程非常简单:
\onslide<+->
    \begin{itemize}
        \item 修改代码 \(\ldots\) 提交修改 \(\ldots\)
        \item 修改代码 \(\ldots\) 提交修改 \(\ldots\)
        \item 修改代码 \(\ldots\) 提交修改 \(\ldots\)
        \item \(\ldots\)
    \end{itemize}
\onslide<+->
    \begin{Verbatim}[frame=single,commandchars=\\\{\}]
git add <file>
git commit
    \end{Verbatim}
\onslide<+->
    \begin{Verbatim}[frame=single,commandchars=\\\{\}]
git commit -a
    \end{Verbatim}
\onslide<+->
    \begin{Verbatim}[frame=single,commandchars=\\\{\}]
git commit -am 'The comment for this revision'
    \end{Verbatim}
\end{frame}

\begin{frame}[fragile]{小结(续)}
\onslide<+->
其他的辅助命令
\onslide<+->
    \begin{Verbatim}[frame=single,commandchars=\\\{\}]
git init
    \end{Verbatim}
\onslide<+->
    \begin{Verbatim}[frame=single,commandchars=\\\{\}]
git status
    \end{Verbatim}
\onslide<+->
    \begin{Verbatim}[frame=single,commandchars=\\\{\}]
git log
    \end{Verbatim}
\onslide<+->

获得在线帮助的方法:
    \begin{Verbatim}[frame=single,commandchars=\\\{\}]
git help
git help commit
git commit --help
    \end{Verbatim}
\end{frame}


\part{分支与合并}

\begin{frame}[<+->][fragile]{新建分支}
\onslide<+->
项目发布后,进入下一阶段开发,不想影响已经发布的版本
\onslide<+->
\begin{Verbatim}[frame=single,commandchars=\\\{\}]
git branch dev
git checkout dev
\end{Verbatim}

已经另建了一个分支,并且转移到新的分支上。

\onslide<+->
\begin{Verbatim}[frame=single,commandchars=\\\{\}]
git branch
\end{Verbatim}

\onslide<+->
修改源程序,并把修改过的文件提交到仓库
\begin{Verbatim}[frame=single,commandchars=\\\{\}]
vi main.c
git add main.c
git commit -m 'Some function is added.'
\end{Verbatim}

这是在新的分支上的一个提交,不影响原来的分支。
\end{frame}

\begin{frame}{新的分支上的提交}
\begin{tikzpicture}
[ commit/.style={ circle, minimum size=11mm, draw=black, },
start chain,node distance=5mm, every node/.style={on chain,join}, every join/.style={<-}
]
\visible<+->{
\foreach \x in {C1,C2}
{ \node [commit] {\x}; }
\node [commit] {C3};
    \begin{scope}[start branch=master]
    \node (master) [on chain=going above, node distance=3mm] {master};
    \end{scope}
\visible<+>{
    \begin{scope}[start branch=dev]
        \node (dev) [on chain=going below, node distance=3mm] {dev};
    \end{scope}}}
\visible<+->{
    \begin{scope}[start branch=plus]
    \node (plus) [commit,on chain=going below right, node distance=8mm] {C4};
        \visible<.>{
        \begin{scope}[start branch=tipdev]
            \node (tipdev) [on chain=going below, node distance=3mm] {dev};
        \end{scope}}
    \visible<+->{\node (plus) [commit] {C5};
        \visible<.>{
        \begin{scope}[start branch=tipdev]
            \node (tipdev) [on chain=going below, node distance=3mm] {dev};
        \end{scope}}
    }
%       \visible<+->{\node (plus) [commit] {C6};
%           \begin{scope}[start branch=tipdev]
%               \node (tipdev) [on chain=going below, node distance=3mm] {dev};
%           \end{scope}
%       }
    \end{scope}}
\end{tikzpicture}
\end{frame}

%   \begin{frame}[fragile]{提交命令的变化}
%   刚才的命令
%   \onslide<+->
%   \begin{Verbatim}[frame=single,commandchars=\\\{\}]
%   git add main.c
%   git add common.c
%   git commit -m 'Some function is added.'
%   \end{Verbatim}
%   
%   \onslide<+->
%   这等同于
%   \begin{Verbatim}[frame=single,commandchars=\\\{\}]
%   git commit -a -m 'Some function is added.'
%   \end{Verbatim}
%   \end{frame}
%   
%   \begin{frame}[<+->][fragile]{在编辑器中输入提交注释}
%   \onslide<+->
%   提交时的注释很重要,也可以不在命令行中输入,而是在一个编辑器中输入。
%   \begin{Verbatim}[frame=single,commandchars=\\\{\}]
%   git commit
%   \end{Verbatim}
%   
%   \onslide<+->
%   缺省编辑器的配置
%   \begin{Verbatim}[frame=single,commandchars=\\\{\}]
%   git config --global core.editor vim
%   \end{Verbatim}
%   \end{frame}

\begin{frame}[<+->][fragile]{切换分支}
\onslide<+->
正在dev分支上工作,有用户报告已经发布的版本中的bug,
无法等到下一个版本发布,需要尽快解决。

\onslide<+->
当前工作先提交
\begin{Verbatim}[frame=single,commandchars=\\\{\}]
git commit -a
\end{Verbatim}

\onslide<+->
切换到master分支,修复bug并提交
\begin{Verbatim}[frame=single,commandchars=\\\{\}]
git checkout master
\(\cdots\)
git commit -a
\end{Verbatim}

\onslide<+->
切换回来,继续dev分支上的开发工作
\begin{Verbatim}[frame=single,commandchars=\\\{\}]
git checkout dev
\(\cdots\)
git commit -a
\end{Verbatim}
\end{frame}

\begin{frame}{分支切换示意}
\begin{tikzpicture}
[ commit/.style={ circle, minimum size=11mm, draw=black, },
start chain,node distance=5mm, every node/.style={on chain,join}, every join/.style={<-}
]
\node [commit] {C2};
\node [commit] {C3};
    \visible<1-3>{
        \begin{scope}[start branch=master]
            \node (master) [on chain=going above, node distance=3mm] {master};
        \end{scope}
    }
    \visible<3>{
        \begin{scope}[start branch=tiphead]
            \node (tiphead) [on chain=going above right, node distance=3mm] {HEAD};
        \end{scope}
    }
    \begin{scope}[start branch=plus]
    \node (plus) [commit,on chain=placed {at=(-60:25mm)}, node distance=11mm] {C4};
    \node (plus) [commit] {C5};
        \visible<1-5>{
            \begin{scope}[start branch=tipdev]
                \node (tipdev) [on chain=going below, node distance=3mm] {dev};
            \end{scope}
        }
        \visible<2,5>{
            \begin{scope}[start branch=tiphead]
                \node (tiphead) [on chain=going below right, node distance=3mm] {HEAD};
            \end{scope}
        }
    \visible<6>{
    \node (plus) [commit] {C7};
        \begin{scope}[start branch=tipdev]
            \node (tipdev) [on chain=going below, node distance=3mm] {dev};
        \end{scope}
        \begin{scope}[start branch=tiphead]
            \node (tiphead) [on chain=going below right, node distance=3mm] {HEAD};
        \end{scope}
    }
    \end{scope}
\visible<4-6>{    
\node [commit] {C6};    
    \begin{scope}[start branch=master]
    \node (master) [on chain=going above, node distance=3mm] {master};
    \end{scope}
    \visible<4>{
        \begin{scope}[start branch=tiphead]
            \node (tiphead) [on chain=going above right, node distance=3mm] {HEAD};
        \end{scope}
    }
}
\end{tikzpicture}
\end{frame}

\begin{frame}[fragile]{分支合并}
\onslide<+->
    master分支内所发现的bug,在目前的dev分支内,也是存在的,
    希望把对那个bug的修复,合并到dev分支中。

\onslide<+->
\begin{Verbatim}[frame=single,commandchars=\\\{\}]
git checkout dev
git merge master
\end{Verbatim}

若合并成功,将会自动产生新的提交。
\onslide<+->
\begin{Verbatim}[frame=single,commandchars=\\\{\}]
git log
\end{Verbatim}
可以观察到
\end{frame}

\begin{frame}{master合并到dev}
\begin{tikzpicture}
[ commit/.style={ circle, minimum size=11mm, draw=black, },
start chain=main, node distance=5mm, every node/.style={on chain,join}, every join/.style={<-}
]
\node [commit] {C2};
\node [commit] {C3};
    \begin{scope}[start branch=plus]
    \node (plus) [commit,on chain=placed {at=(-60:25mm)}, node distance=11mm] {C4};
    \node (plus) [commit] {C5};
    \node (plus) [commit] {C7};
        \visible<1>{
        \begin{scope}[start branch=tipdev]
            \node (tipdev) [on chain=going below, node distance=3mm] {dev};
        \end{scope}
        }
    \end{scope}
\node [commit] {C6};    
    \begin{scope}[start branch=master]
    \node (master) [on chain=going above, node distance=3mm] {master};
    \end{scope}
    \visible<2->{
    \begin{scope}[continue branch=plus]
    \node (plus) [commit, join=with main-3] {C8};
        \begin{scope}[start branch=tipdev]
            \node (tipdev) [on chain=going below, node distance=3mm] {dev};
        \end{scope}
    \end{scope}
    }
\end{tikzpicture}
\end{frame}

\begin{frame}[<+->][fragile]{更加专业的做法}
    \begin{itemize}
        \item Git分支与合并的成本很低
        \item 在新的分支上修改,然后合并回来的做法,更加专业
        \item 修复新发现bug的通行做法是
        \begin{Verbatim}[frame=single,commandchars=\\\{\}]
git checkout master
git branch fixbug
git checkout fixbug
# Fix bugs ...
git add ...
git commit
# Test
git checkout master
git merge fixbug
git branch -d fixbug
        \end{Verbatim}
        \item 上面最后一条是删除分支的命令
    \end{itemize}
\end{frame}

\begin{frame}{在新建的修复分支中处理缺陷}
\begin{tikzpicture}
[ commit/.style={ circle, minimum size=11mm, draw=black, },
start chain,node distance=5mm, every node/.style={on chain}, every join/.style={<-}
]
\node [commit, join] {C2};
\node [commit, join] (c3) {C3};
    \visible<1-7,12>{
        \begin{scope}[start branch=master]
            \node [on chain=going above, node distance=3mm, join] {master};
        \end{scope}
    }
    \visible<2-3,7>{
        \begin{scope}[start branch=tiphead]
            \node [on chain=going right, node distance=3mm, join] {HEAD};
        \end{scope}
    }
    \visible<3>{
        \begin{scope}[start branch=tipfix]
            \node [on chain=going below, node distance=3mm, join] {fixbug};
        \end{scope}
    }
    \begin{scope}[start branch=plus]
    \node [commit, join,on chain=going below right, node distance=11mm] {C4};
    \node [commit, join] {C5};
            \begin{scope}[start branch=tipdev]
                \node [on chain=going below, node distance=3mm, join] {dev};
            \end{scope}
        \visible<1>{
            \begin{scope}[start branch=tiphead]
                \node [on chain=going right, node distance=3mm, join] {HEAD};
            \end{scope}
        }
    \end{scope}
    
\begin{scope}    
\visible<4-9,12->{
\node [commit, join, on chain=going above right, node distance=11mm] {C6};    
}
    \visible<4>{
        \begin{scope}[start branch=master]
            \node [on chain=going below, node distance=3mm, join] {fixbug};
        \end{scope}
        \begin{scope}[start branch=tiphead]
            \node [on chain=going right, node distance=3mm, join] {HEAD};
        \end{scope}
    }
\visible<5-9,12->{
\node [commit, join] {C7};    
}
    \visible<5>{
        \begin{scope}[start branch=master]
            \node [on chain=going below, node distance=3mm, join] {fixbug};
        \end{scope}
        \begin{scope}[start branch=tiphead]
            \node [on chain=going right, node distance=3mm, join] {HEAD};
        \end{scope}
    }
\visible<6-9,12->{
\node [commit, join] {C8};    
}
    \visible<6-8,12-13>{
        \begin{scope}[start branch=master]
            \node [on chain=going below, node distance=3mm, join] {fixbug};
        \end{scope}
    }
    \visible<6,8-9,12>{
        \begin{scope}[start branch=tiphead]
            \node [on chain=going right, node distance=3mm, join] {HEAD};
        \end{scope}
    }
    \visible<8-9>{
        \begin{scope}[start branch=master]
            \node [on chain=going above, node distance=3mm, join] {master};
        \end{scope}
    }
\visible<13->{
\node [commit, very thick, red, join, on chain=going below right, node distance=11mm, join=with c3] {C9};    
    \begin{scope}[start branch=master]
    \node [on chain=going above, node distance=3mm, join] {master};
    \end{scope}
        \begin{scope}[start branch=tiphead]
            \node [on chain=going right, node distance=3mm, join] {HEAD};
        \end{scope}
}
\end{scope}

\visible<10>{
\node [commit, above right=0mm and 8mm of c3, join=with c3] {C6};    
\node [commit, join] {C7};    
\node [commit, join] {C8};    
        \begin{scope}[start branch=tiphead]
            \node [on chain=going right, node distance=3mm, join] {HEAD};
        \end{scope}
        \begin{scope}[start branch=master]
            \node [on chain=going above, node distance=3mm, join] {master};
        \end{scope}
}

\visible<11>{
\node [commit, right=of c3, join=with c3] {C6};    
\node [commit, join] {C7};    
\node [commit, join] {C8};    
        \begin{scope}[start branch=tiphead]
            \node [on chain=going right, node distance=3mm, join] {HEAD};
        \end{scope}
        \begin{scope}[start branch=master]
            \node [on chain=going above, node distance=3mm, join] {master};
        \end{scope}
}

\end{tikzpicture}
\end{frame}

\begin{frame}[<+->][fragile]{更简洁的命令}
    \begin{itemize}
        \item 创建新分支并切换过去
        \begin{Verbatim}[frame=single,commandchars=\\\{\}]
git checkout -b fixbug      
        \end{Verbatim}
        \item 被修改的文件,加入暂存区并提交
        \begin{Verbatim}[frame=single,commandchars=\\\{\}]
git commit -a
        \end{Verbatim}
        \item 切换到新分支的同时,把当前修改过的文件一起合并过去
        \begin{Verbatim}[frame=single,commandchars=\\\{\}]
git checkout -m master
        \end{Verbatim}
        要注意合并是否出现冲突
    \end{itemize}
\end{frame}


\begin{frame}{dev合并到master}
\begin{tikzpicture}
[ commit/.style={ circle, minimum size=11mm, draw=black, },
start chain=main, node distance=5mm, every node/.style={on chain,join}, every join/.style={<-}
]
\node [commit] {C3};
    \begin{scope}[start branch=dev]
    \node [commit,on chain=placed {at=(-90:25mm)}, node distance=11mm] {C4};
    \node [commit] {C5};
    \node [commit] {C7};
    \end{scope}
\node [commit] {C6};    
    \visible<1-2>{
    \begin{scope}[start branch]
    \node [on chain=going above, node distance=3mm] {master};
    \end{scope}
    }
    \begin{scope}[continue branch=dev]
    \node [commit, join=with main-2] {C8};
        \visible<1>{
        \begin{scope}[start branch]
            \node [on chain=going below, node distance=3mm] {dev};
        \end{scope}
        }
    \visible<2->{
    \node (c9) [commit] {C9};
        \begin{scope}[start branch]
            \node (tipdev) [on chain=going below, node distance=3mm] {dev};
        \end{scope}
    }
    \end{scope}
\visible<3->{
\node [commit, on chain=going right, node distance=60mm, join=with c9 ] {C10};    
    \begin{scope}[start branch]
        \node [on chain=going above, node distance=3mm] {master};
    \end{scope}
}
\end{tikzpicture}
\end{frame}

\begin{frame}[<+->][fragile]{分支合并冲突}
\onslide<+->
不同分支上,对同一部分的修改不一致,合并时会产生冲突
\begin{Verbatim}[frame=single,commandchars=\\\{\}]
$ git merge dev
Auto-merging main.c
CONFLICT (content): Merge conflict in main.c
Automatic merge failed; fix conflicts 
and then commit the result.
\end{Verbatim}

\onslide<+->
合并发生冲突时,不会自动产生新的提交。
\end{frame}

\begin{frame}[<+->][fragile]{分支合并冲突(续)}
\onslide<+->
产生冲突的文件内容变成
\begin{Verbatim}[frame=single,commandchars=\@\[\]]
#include<stdio.h>
int main()
{
        printf("Hello, World!\n");
<<<<<<< HEAD
        printf("This is second line.\n");
=======
        printf("The 2nd line is here.\n");
>>>>>>> dev
}
\end{Verbatim}
\end{frame}

\begin{frame}[fragile]{解决合并时的冲突}
\onslide<+->
手工解决冲突,然后提交
\begin{Verbatim}[frame=single,commandchars=\\\{\}]
vi main.c
\(\cdots\)
git add main.c
git commit
\end{Verbatim}
\end{frame}

\begin{frame}{小结}
人们在使用Git作为源代码控制的情况,绝大多数工作流程如下:
    \begin{itemize}
        \item 修改代码 \(\ldots\) 提交修改 \(\ldots\)
        \item 创建分支,切换分支
        \item 修改代码 \(\ldots\) 提交修改 \(\ldots\)
        \item 切换分支
        \item 修改代码 \(\ldots\) 提交修改 \(\ldots\)
        \item 切换分支
        \item 合并分支
        \item \(\ldots\)
    \end{itemize}
\end{frame}
