\part{更熟悉一点Git的使用}
\begin{frame}[<+->][fragile]{分支/合并是Git的核心能力}
    \begin{itemize}
        \item 新建分支,切换分支,很容易
\begin{Verbatim}[frame=single,commandchars=\\\{\}]
git branch newfeature
git checkout newfeature
\end{Verbatim}
        \item 两步合一步
\begin{Verbatim}[frame=single,commandchars=\\\{\}]
git checkout -b newfeature            
\end{Verbatim}
        \item 新建分支不是大事儿
        \item 习惯的做法:

新建临时分支并切换到那里,修改代码,提交,再修改,再提交 \(\ldots\) 回到原分支,将临时分支合并过来,删除 \(\ldots\)临时分支
    \end{itemize}
\end{frame}

\begin{frame}
    \includegraphics<1>[width=9cm]{figure/branch-merge-cycle-1.png}
    \includegraphics<2>[width=9cm]{figure/branch-merge-cycle-1-G.png}
    \includegraphics<3>[width=9cm]{figure/branch-merge-cycle-1-H.png}
    \includegraphics<4>[width=9cm]{figure/branch-merge-cycle-1-I.png}
    \includegraphics<5>[width=9cm]{figure/branch-merge-cycle-1-J.png}
    \includegraphics<6>[width=9cm]{figure/branch-merge-cycle-1-K.png}
    \includegraphics<7>[width=9cm]{figure/branch-merge-cycle-1-L.png}
    \includegraphics<8>[width=9cm]{figure/branch-merge-cycle-2.png}
    \includegraphics<9>[width=9cm]{figure/branch-merge-cycle-3.png}
    \includegraphics<10>[width=9cm]{figure/branch-merge-cycle-4.png}
\end{frame}

\begin{frame}[<+->][fragile]{删除分支}
    \begin{itemize}
        \item 删除分支,很容易
\begin{Verbatim}[frame=single,commandchars=\\\{\}]
git branch -d newfeature
\end{Verbatim}
        \item 只有已经合并的分支才能删除
        \item 删除分支不是大事儿
    \end{itemize}
\end{frame}

\begin{frame}[<+->][fragile]{提交的历史}
    \begin{itemize}
        \item 每一次提交都被保存在仓库中,用一串sha1校验和表示
        \item 观察每一次的提交:
\begin{Verbatim}[frame=single,commandchars=\\\{\}]
git log
\end{Verbatim}
        \item 或者
\begin{Verbatim}[frame=single,commandchars=\\\{\}]
git log --pretty=oneline
\end{Verbatim}
        \item 分别得到这样的结果
    \end{itemize}
\end{frame}

\begin{frame}[<+->][fragile]{}
    \begin{itemize}
        \item 分别得到这样的结果
\begin{Verbatim}[frame=single,commandchars=\\\{\}]
commit 81f5046656fc04665f7a5224cc07cfbbdb14baf6
Author: Wang Shaodong <wsd@wsd-hp>
Date:   Wed Mar 14 07:10:36 2012 +0800

    remove a branch

commit 1a3a3437775f20b1caa6dfa41d22539a17db542e
Author: Wang Shaodong <wsd@wsd-hp>
Date:   Wed Mar 14 07:03:44 2012 +0800

    file .gitignore is added
\end{Verbatim}

        \item 或者

\begin{Verbatim}[frame=single,commandchars=\\\{\}]
81f5046656fc04665f7a5224cc07cfbbdb14baf6 remove a branch
1a3a3437775f20b1caa6dfa41d22539a17db542e file .gitignore is added
236c0268eaddc68c8a03a867a482a56317f2649a branch-merge-cycle
533e3140bffee43b02c5648c8fcc3e63232739a6 fix a bug on figure resolve-conflict
\end{Verbatim}
    \end{itemize}
\end{frame}

\begin{frame}[<+->][fragile]{标识每一个提交}
    \begin{itemize}
        \item 绝对提交名:533e3140bffee43b02c5648c8fcc3e63232739a6
        \item 绝对提交名的简写:533e31
        \item 参照名:remotes/origin/master, newfeature
        \item 符号参照名:\verb|HEAD, ORIG_HEAD, FETCH_HEAD, MERGE_HEAD|
        \item 相对提交名:\verb|master^, master^^, master~2, master~10^2|
    \end{itemize}
\end{frame}

\begin{frame}
    \includegraphics<1>[width=9cm]{figure/commit-relative-name-1.png}
    \includegraphics<2>[width=9cm]{figure/commit-relative-name-2.png}
    \includegraphics<3>[width=9cm]{figure/commit-relative-name-3.png}
    \includegraphics<4>[width=9cm]{figure/commit-relative-name-4.png}
    \includegraphics<5>[width=9cm]{figure/commit-relative-name-5.png}
\end{frame}

\begin{frame}[<+->][fragile]{}
    \begin{itemize}
        \item 尝试
\begin{Verbatim}[frame=single,commandchars=\\\{\}]
git rev-parse master
\end{Verbatim}
        \item 再尝试
\begin{Verbatim}[frame=single,commandchars=\\\{\}]
git show-branch --more=35 --all
\end{Verbatim}
        \item 还可以试试
\begin{Verbatim}[frame=single,commandchars=\\\{\}]
git rev-parse master~3
\end{Verbatim}
    \end{itemize}
\end{frame}

\begin{frame}[<+->][fragile]{穿越:回到从前}
    \begin{itemize}
        \item 要检查某个历史上的版本
\begin{Verbatim}[frame=single,commandchars=\\\{\}]
git checkout 533e31
\end{Verbatim}
        \item 或者
\begin{Verbatim}[frame=single,commandchars=\\\{\}]
git checkout master~3
\end{Verbatim}
        \item 还可以试试
\begin{Verbatim}[frame=single,commandchars=\\\{\}]
git checkout :/"some strings to find in comment"
\end{Verbatim}
        \item 回到当下
\begin{Verbatim}[frame=single,commandchars=\\\{\}]
git checkout youbranchname
\end{Verbatim}
    \end{itemize}
\end{frame}
