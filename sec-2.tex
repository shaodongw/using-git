\part{多Git一点}

\begin{frame}[<+->][fragile]{查看分支}
    \begin{itemize}
        \item 当前所在分支
        \begin{Verbatim}[frame=single,commandchars=\\\{\}]
git branch        
        \end{Verbatim}
        \item 了解分支情况
        \begin{Verbatim}[frame=single,commandchars=\\\{\}]
git show-branch
        \end{Verbatim}
        \item 绘出分支历史关系图
        \begin{Verbatim}[frame=single,commandchars=\\\{\}]
git log --graph --all --pretty=oneline
        \end{Verbatim}
    \end{itemize}
\end{frame}

\begin{frame}[<+->][fragile]{观察历史}
    \begin{itemize}
        \item 每一次提交都被保存在仓库中,用一串sha1校验和表示
        \item 观察每一次的提交:
\begin{Verbatim}[frame=single,commandchars=\\\{\}]
git log
\end{Verbatim}
        \item 或者
\begin{Verbatim}[frame=single,commandchars=\\\{\}]
git log --pretty=oneline
\end{Verbatim}
    \end{itemize}
\end{frame}

\begin{frame}[<+->][fragile]{标识历史上的每一个提交}
    \begin{itemize}
        \item 绝对提交名:533e3140bffee43b02c5648c8fcc3e63232739a6
        \item 绝对提交名的简写:533e31
        \item 参照名:master, dev, fixbug, v1.0, remotes/origin/master
        \item 符号参照名:\verb|HEAD, ORIG_HEAD, FETCH_HEAD, MERGE_HEAD|
        \item 相对提交名:\verb|master^, master^^, master~2, master~10^2|
    \end{itemize}
\end{frame}

\begin{frame}[<+->][fragile]{相对提交名}
    \begin{itemize}
        \item 尝试
\begin{Verbatim}[frame=single,commandchars=\\\{\}]
git rev-parse master
\end{Verbatim}
        \item 尝试
\begin{Verbatim}[frame=single,commandchars=\\\{\}]
git show-branch --more=3 --all
\end{Verbatim}
        \item 尝试
\begin{Verbatim}[frame=single,commandchars=\\\{\}]
git rev-parse master~2
\end{Verbatim}
        \item 尝试
\begin{Verbatim}[frame=single,commandchars=\\\{\}]
git rev-parse HEAD^
git rev-parse HEAD^^
git rev-parse HEAD^2
\end{Verbatim}
    \end{itemize}
\end{frame}

\begin{frame}[<+->][fragile]{标签}
    \begin{itemize}
        \item 给当前的版本加标签
        \begin{Verbatim}[frame=single,commandchars=\\\{\}]
git tag -a v1.0
        \end{Verbatim}
        \item 查看标签
        \begin{Verbatim}[frame=single,commandchars=\\\{\}]
git tag
        \end{Verbatim}
        \item 了解标签所对应的提交名
        \begin{Verbatim}[frame=single,commandchars=\\\{\}]
git rev-parse v1.0
        \end{Verbatim}
        \item 在提交历史中显示标签
        \begin{Verbatim}[frame=single,commandchars=\\\{\}]
git log --decorate
        \end{Verbatim}
    \end{itemize}
\end{frame}

\begin{frame}[<+->][fragile]{标签(续)}
    \begin{itemize}
        \item 给历史上的版本加标签
        \begin{Verbatim}[frame=single,commandchars=\\\{\}]
git checkout fd311e
git tag -a v1.0 -m 'first release'
git checkout dev
        \end{Verbatim}
        \item 或者
        \begin{Verbatim}[frame=single,commandchars=\\\{\}]
git tag -a v1.1 9feb0 -m 'improved version'
        \end{Verbatim}
        \item 轻量标签
        \begin{Verbatim}[frame=single,commandchars=\\\{\}]
git tag birth-day-revision
        \end{Verbatim}
    \end{itemize}
\end{frame}

\begin{frame}[<+->][fragile]{穿越:回到从前}
    \begin{itemize}
        \item 要检查某个历史上的版本
\begin{Verbatim}[frame=single,commandchars=\\\{\}]
git checkout 533e31
\end{Verbatim}
        \item 或者
\begin{Verbatim}[frame=single,commandchars=\\\{\}]
git checkout v1.0
git checkout master~3
\end{Verbatim}
        \item 试试
\begin{Verbatim}[frame=single,commandchars=\\\{\}]
git checkout :/"some strings to find in comment"
\end{Verbatim}
        \item 回到当下
\begin{Verbatim}[frame=single,commandchars=\\\{\}]
git checkout dev
\end{Verbatim}
    \end{itemize}
\end{frame}

\begin{frame}[<+->][fragile]{基于历史版本的修改}
    \begin{itemize}
        \item 'detached HEAD' state 概念
        \item 查看/修改/试验/提交/抛弃
        \item 另建分支以便将来可以找到它
\begin{Verbatim}[frame=single,commandchars=\\\{\}]
\(\cdots\)
git checkout -b new_branch_name
\(\cdots\)
git commit
git checkout dev
\end{Verbatim}
    \end{itemize}
\end{frame}

\begin{frame}[<+->][fragile]{干净的工作目录}
    \begin{itemize}
        \item 已修改未更新
        \item 已更新待提交
        \item 提交后又修改
        \item 不干净的工作目录,妨碍你切换到其他版本
    \end{itemize}
\end{frame}

\begin{frame}[<+->][fragile]{抛弃与撤销}
    \begin{itemize}
        \item 抛弃所做的修改,强制切换
        \begin{Verbatim}[frame=single,commandchars=\\\{\}]
git checkout -f dev
        \end{Verbatim}
        \item 撤销所做的修改,留在当前分支
        \begin{Verbatim}[frame=single,commandchars=\\\{\}]
git checkout -- main.c
        \end{Verbatim}
        \item 修改已经更新到暂存区,尚未提交
        \begin{Verbatim}[frame=single,commandchars=\\\{\}]
git reset HEAD main.c
git checkout -- main.c
        \end{Verbatim}
        \item 两步合并成一步
        \begin{Verbatim}[frame=single,commandchars=\\\{\}]
git reset --hard HEAD main.c
        \end{Verbatim}
    \end{itemize}
\end{frame}

