\part{更熟悉一点Git的使用}
\begin{frame}[<+->][fragile]{分支/合并是Git的核心能力}
    \begin{itemize}
        \item 新建分支,切换分支,很容易
\begin{Verbatim}[frame=single,commandchars=\\\{\}]
git branch newfeature
git checkout newfeature
\end{Verbatim}
        \item 两步合一步
\begin{Verbatim}[frame=single,commandchars=\\\{\}]
git checkout -b newfeature            
\end{Verbatim}
        \item 新建分支不是大事儿
        \item 习惯的做法:

新建临时分支并切换到那里,修改代码,提交,再修改,再提交 \(\ldots\) 回到原分支,将临时分支合并过来,删除 \(\ldots\)临时分支
    \end{itemize}
\end{frame}

\begin{frame}
    \includegraphics<1>[width=9cm]{figure/branch-merge-cycle-1.png}
    \includegraphics<2>[width=9cm]{figure/branch-merge-cycle-1-G.png}
    \includegraphics<3>[width=9cm]{figure/branch-merge-cycle-1-H.png}
    \includegraphics<4>[width=9cm]{figure/branch-merge-cycle-1-I.png}
    \includegraphics<5>[width=9cm]{figure/branch-merge-cycle-1-J.png}
    \includegraphics<6>[width=9cm]{figure/branch-merge-cycle-1-K.png}
    \includegraphics<7>[width=9cm]{figure/branch-merge-cycle-1-L.png}
    \includegraphics<8>[width=9cm]{figure/branch-merge-cycle-2.png}
    \includegraphics<9>[width=9cm]{figure/branch-merge-cycle-3.png}
    \includegraphics<10>[width=9cm]{figure/branch-merge-cycle-4.png}
\end{frame}

\begin{frame}[<+->][fragile]{删除分支}
    \begin{itemize}
        \item 删除分支,很容易
\begin{Verbatim}[frame=single,commandchars=\\\{\}]
git branch -d newfeature
\end{Verbatim}
        \item 只有已经合并的分支才能删除
        \item 删除分支不是大事儿
    \end{itemize}
\end{frame}

