\part{多Git一点}

\begin{frame}[<+->][fragile]{查看分支}
    \begin{itemize}
        \item 当前所在分支
        \begin{Verbatim}[frame=single,commandchars=\\\{\}]
git branch        
        \end{Verbatim}
        \item 了解分支情况
        \begin{Verbatim}[frame=single,commandchars=\\\{\}]
git show-branch
        \end{Verbatim}
        \item 绘出分支历史关系图
        \begin{Verbatim}[frame=single,commandchars=\\\{\}]
git log --graph --all --pretty=oneline
        \end{Verbatim}
    \end{itemize}
\end{frame}

\begin{frame}[<+->][fragile]{观察历史}
    \begin{itemize}
        \item 每一次提交都被保存在仓库中,用一串sha1校验和表示
        \item 观察每一次的提交:
\begin{Verbatim}[frame=single,commandchars=\\\{\}]
git log
\end{Verbatim}
        \item 或者
\begin{Verbatim}[frame=single,commandchars=\\\{\}]
git log --pretty=oneline
\end{Verbatim}
    \end{itemize}
\end{frame}

\begin{frame}[<+->][fragile]{标识历史上的每一个提交}
    \begin{itemize}
        \item 绝对提交名:533e3140bffee43b02c5648c8fcc3e63232739a6
        \item 绝对提交名的简写:533e31
        \item 参照名:master, dev, fixbug, v1.0, remotes/origin/master
        \item 符号参照名:\verb|HEAD, ORIG_HEAD, FETCH_HEAD, MERGE_HEAD|
        \item 相对提交名:\verb|master^, master^^, master~2, master~10^2|
    \end{itemize}
\end{frame}

\begin{frame}[<+->][fragile]{相对提交名}
    \begin{itemize}
        \item 尝试
\begin{Verbatim}[frame=single,commandchars=\\\{\}]
git rev-parse master
\end{Verbatim}
        \item 尝试
\begin{Verbatim}[frame=single,commandchars=\\\{\}]
git show-branch --more=3 --all
\end{Verbatim}
        \item 尝试
\begin{Verbatim}[frame=single,commandchars=\\\{\}]
git rev-parse master~2
\end{Verbatim}
        \item 尝试
\begin{Verbatim}[frame=single,commandchars=\\\{\}]
git rev-parse HEAD^
git rev-parse HEAD^^
git rev-parse HEAD^2
\end{Verbatim}
    \end{itemize}
\end{frame}

\begin{frame}[<+->][fragile]{标签}
    \begin{itemize}
        \item 给当前的版本加标签
        \begin{Verbatim}[frame=single,commandchars=\\\{\}]
git tag -a v1.0
        \end{Verbatim}
        \item 查看标签
        \begin{Verbatim}[frame=single,commandchars=\\\{\}]
git tag
        \end{Verbatim}
        \item 了解标签所对应的提交名
        \begin{Verbatim}[frame=single,commandchars=\\\{\}]
git rev-parse v1.0
        \end{Verbatim}
        \item 在提交历史中显示标签
        \begin{Verbatim}[frame=single,commandchars=\\\{\}]
git log --decorate
        \end{Verbatim}
    \end{itemize}
\end{frame}

\begin{frame}[<+->][fragile]{标签(续)}
    \begin{itemize}
        \item 给历史上的版本加标签
        \begin{Verbatim}[frame=single,commandchars=\\\{\}]
git checkout fd311e
git tag -a v1.0 -m 'first release'
git checkout dev
        \end{Verbatim}
        \item 或者
        \begin{Verbatim}[frame=single,commandchars=\\\{\}]
git tag -a v1.1 9feb0 -m 'improved version'
        \end{Verbatim}
        \item 轻量标签
        \begin{Verbatim}[frame=single,commandchars=\\\{\}]
git tag birth-day-revision
        \end{Verbatim}
    \end{itemize}
\end{frame}

\begin{frame}[<+->][fragile]{穿越:回到从前}
    \begin{itemize}
        \item 要检查某个历史上的版本
\begin{Verbatim}[frame=single,commandchars=\\\{\}]
git checkout 533e31
\end{Verbatim}
        \item 或者
\begin{Verbatim}[frame=single,commandchars=\\\{\}]
git checkout v1.0
git checkout master~3
\end{Verbatim}
        \item 试试
\begin{Verbatim}[frame=single,commandchars=\\\{\}]
git checkout :/"some strings to find in comment"
\end{Verbatim}
        \item 回到当下
\begin{Verbatim}[frame=single,commandchars=\\\{\}]
git checkout dev
\end{Verbatim}
    \end{itemize}
\end{frame}

\begin{frame}[<+->][fragile]{基于历史版本的修改}
    \begin{itemize}
        \item 'detached HEAD' state 概念
        \item 查看/修改/试验/提交/抛弃
        \item 另建分支以便将来可以找到它
\begin{Verbatim}[frame=single,commandchars=\\\{\}]
\(\cdots\)
git checkout -b new_branch_name
\(\cdots\)
git commit
git checkout dev
\end{Verbatim}
    \end{itemize}
\end{frame}

\begin{frame}[<+->][fragile]{干净的工作目录}
    \begin{itemize}
        \item 已修改未更新
        \item 已更新待提交
        \item 提交后又修改
        \item 不干净的工作目录,妨碍你切换到其他版本
    \end{itemize}
\end{frame}

\begin{frame}[<+->][fragile]{抛弃与撤销}
    \begin{itemize}
        \item 抛弃所做的修改,强制切换
        \begin{Verbatim}[frame=single,commandchars=\\\{\}]
git checkout -f dev
        \end{Verbatim}
        \item 撤销所做的修改,留在当前分支
        \begin{Verbatim}[frame=single,commandchars=\\\{\}]
git checkout -- main.c
        \end{Verbatim}
        \item 修改已经更新到暂存区,尚未提交
        \begin{Verbatim}[frame=single,commandchars=\\\{\}]
git reset HEAD main.c
git checkout -- main.c
        \end{Verbatim}
        \item 两步合并成一步
        \begin{Verbatim}[frame=single,commandchars=\\\{\}]
git reset --hard HEAD main.c
        \end{Verbatim}
    \end{itemize}
\end{frame}

\begin{frame}{示意图}
\begin{tikzpicture}[
commit/.style={circle, minimum size=18mm, draw},
stage/.style={ellipse, minimum size=12mm, draw},
work/.style={rectangle, minimum size=12mm, draw},
>=latex
]
  \node[commit] (c6) {Commit-6};
  \node[commit] (c7) [right=of c6] {Commit-7}
    edge [->] (c6);
  \node[commit] (c8) [right=of c7] {Commit-8}
    edge [->] (c7);

  \node[stage] (stage) [below=15mm of c8] {Staging Area}
    edge [<-, bend left=45] node[auto] {\tt git reset -{}- files} (c8)
    edge [->, bend right=45] node[auto,swap]{\tt git commit} (c8);
  
  \node[work] (work) [below=18mm of stage] {Working Directory}
    edge [<-, bend left=45] node[auto] {\tt git checkout -{}- files} (stage)
    edge [->, bend right=45] node[auto,swap] {\tt git add files} (stage);

    \begin{pgfonlayer}{background}
    \node [fill=black!10,fit=(c6) (c8)] {};
    \end{pgfonlayer}

\end{tikzpicture}
\end{frame}

\begin{frame}[<+->][fragile]{git checkout操作的多义性}
    \begin{itemize}
        \item 切换分支
        \begin{Verbatim}[frame=single,commandchars=\\\{\}]
git checkout branch_name
        \end{Verbatim}
        \item 单纯改变工作目录
        \begin{Verbatim}[frame=single,commandchars=\\\{\}]
git checkout -- main.c
        \end{Verbatim}
        \item 区别:参数是“提交名”,还是“文件名”
        \item 应用范例:“拎出”特定文件的历史版本到当前工作目录
        \begin{Verbatim}[frame=single,commandchars=\\\{\}]
git checkout 7fba82 main.c
git checkout fixbug~4 main.c
git checkout v1.1   main.c
        \end{Verbatim}
    \end{itemize}
\end{frame}


\begin{frame}[<+->][fragile]{逆转与回退}
    \begin{itemize}
        \item 逆转(回滚)
        \begin{Verbatim}[frame=single,commandchars=\\\{\}]
git revert HEAD
        \end{Verbatim}
        \item 回退
        \begin{Verbatim}[frame=single,commandchars=\\\{\}]
git reset --hard HEAD~1
        \end{Verbatim}
        \item 已经向外公布的提交,不应该改变其历史
    \end{itemize}
\end{frame}

\begin{frame}{图示}
\begin{tikzpicture}
[ commit/.style={ circle, minimum size=11mm, draw=black, },
start chain,node distance=5mm, every node/.style={on chain,join}, every join/.style={<-}
]
\node [commit] {C2};
\node [commit] {C3};
    \begin{scope}[start branch=master]
    \node [on chain=going above, node distance=3mm] {master};
    \end{scope}
    \node [commit,on chain=going below right, node distance=8mm] {C4};
        \visible<1,5>{
        \begin{scope}[start branch=tipdev]
            \node [on chain=going below, node distance=3mm] {dev};
        \end{scope}
        }
    \begin{scope}[start branch=plus]
        \visible<2->{
        \node [commit, very thick, red] {C5};
        }
            \visible<2,4>{
            \begin{scope}[start branch=tipdev]
                \node [on chain=going below, node distance=3mm] {dev};
            \end{scope}
            }
        \visible<3>{
        \node [commit] {C4a};
            \begin{scope}[start branch=tipdev]
                \node [on chain=going below, node distance=3mm] {dev};
            \end{scope}
        }
    \end{scope}
    \visible<6->{
    \node [commit, on chain=going below right, node distance=11mm] {C6};
    }
        \visible<6>{
        \begin{scope}[start branch=tipdev]
            \node [on chain=going below, node distance=3mm] {dev};
        \end{scope}
        }
    \visible<7->{
    \node [commit] {C7};
        \begin{scope}[start branch=tipdev]
            \node [on chain=going below, node distance=3mm] {dev};
        \end{scope}
    }
\end{tikzpicture}
\end{frame}

\begin{frame}[<+->][fragile]{再谈合并}
    \begin{itemize}
        \item “快进型”合并,只移动分支指针
        \item 一般的常规合并,自动向仓库产生新的提交
        \item 合并有冲突时,不产生提交,等待处理
        \begin{Verbatim}[frame=single,commandchars=\\\{\}]
# Edit the file \(\ldots\)
git add main.c
git commit
        \end{Verbatim}
        冲突解决后的git commit操作,产生新的提交
    \end{itemize}
\end{frame}

\begin{frame}[<+->][fragile]{对合并的撤销}
    \begin{itemize}
        \item 合并成功,提交已自动产生,想要撤销它
        \begin{Verbatim}[frame=single,commandchars=\\\{\}]
git reset --hard ORIG_HEAD
        \end{Verbatim}
        \item 合并有冲突,提交未产生,此刻不想解决冲突,想要撤销合并
        \begin{Verbatim}[frame=single,commandchars=\\\{\}]
git reset --hard HEAD
        \end{Verbatim}
        \item 合并遇冲突,试图解决,但不满意,重回合并冲突时的情景
        \begin{Verbatim}[frame=single,commandchars=\\\{\}]
git checkout -m
        \end{Verbatim}
    \end{itemize}
\end{frame}
