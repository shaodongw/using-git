\part{多Git一点}

\begin{frame}[<+->][fragile]{查看分支}
    \begin{itemize}
        \item 当前所在分支
        \begin{Verbatim}[frame=single,commandchars=\\\{\}]
git branch        
        \end{Verbatim}
        \item 了解分支情况
        \begin{Verbatim}[frame=single,commandchars=\\\{\}]
git show-branch
        \end{Verbatim}
        \item 绘出分支历史关系图
        \begin{Verbatim}[frame=single,commandchars=\\\{\}]
git log --graph --all --pretty=oneline
        \end{Verbatim}
    \end{itemize}
\end{frame}

\begin{frame}[<+->][fragile]{观察历史}
    \begin{itemize}
        \item 每一次提交都被保存在仓库中,用一串sha1校验和表示
        \item 观察每一次的提交:
\begin{Verbatim}[frame=single,commandchars=\\\{\}]
git log
\end{Verbatim}
        \item 或者
\begin{Verbatim}[frame=single,commandchars=\\\{\}]
git log --pretty=oneline
\end{Verbatim}
    \end{itemize}
\end{frame}

\begin{frame}[<+->][fragile]{标识历史上的每一个提交}
    \begin{itemize}
        \item 绝对提交名:533e3140bffee43b02c5648c8fcc3e63232739a6
        \item 绝对提交名的简写:533e31
        \item 参照名:master, dev, fixbug, v1.0, remotes/origin/master
        \item 符号参照名:\verb|HEAD, ORIG_HEAD, FETCH_HEAD, MERGE_HEAD|
        \item 相对提交名:\verb|master^, master^^, master~2, master~10^2|
    \end{itemize}
\end{frame}

\begin{frame}[<+->][fragile]{相对提交名}
    \begin{itemize}
        \item 尝试
\begin{Verbatim}[frame=single,commandchars=\\\{\}]
git rev-parse master
\end{Verbatim}
        \item 尝试
\begin{Verbatim}[frame=single,commandchars=\\\{\}]
git show-branch --more=3 --all
\end{Verbatim}
        \item 尝试
\begin{Verbatim}[frame=single,commandchars=\\\{\}]
git rev-parse master~2
\end{Verbatim}
        \item 尝试
\begin{Verbatim}[frame=single,commandchars=\\\{\}]
git rev-parse HEAD^
git rev-parse HEAD^^
git rev-parse HEAD^2
\end{Verbatim}
    \end{itemize}
\end{frame}

\begin{frame}[<+->][fragile]{穿越:回到从前}
    \begin{itemize}
        \item 要检查某个历史上的版本
\begin{Verbatim}[frame=single,commandchars=\\\{\}]
git checkout 533e31
\end{Verbatim}
        \item 或者
\begin{Verbatim}[frame=single,commandchars=\\\{\}]
git checkout master~3
\end{Verbatim}
        \item 还可以试试
\begin{Verbatim}[frame=single,commandchars=\\\{\}]
git checkout :/"some strings to find in comment"
\end{Verbatim}
        \item 回到当下
\begin{Verbatim}[frame=single,commandchars=\\\{\}]
git checkout youbranchname
\end{Verbatim}
    \end{itemize}
\end{frame}
