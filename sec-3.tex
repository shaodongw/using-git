\part{多人协同工作}
\begin{frame}[<+->][fragile]{初始工作}
\begin{itemize}
  \item 克隆操作
\begin{Verbatim}[frame=single,commandchars=\\\{\}]
git clone ssh://example.com/git/our-project.git
git clone git@example.com:git/our-project.git
\end{Verbatim}
  \item 远程仓库
  \item 本地仓库
  \begin{enumerate}
    \item 工作目录(缺省:master分支的最新本)
    \item 内含本地仓库(\ .git子目录)
  \end{enumerate}
  \item Development Repository \& Bare Repository
\end{itemize}
\end{frame}

\begin{frame}[<+->][fragile]{本地仓库}
\begin{itemize}
  \item 本地仓库一旦建立,操作就针对本地仓库
\begin{Verbatim}[frame=single,commandchars=\\\{\}]
git log
\end{Verbatim}
  \item 可以看到所有的提交历史,都被复制过来。
  \item 自动为你建立工作目录,缺省的分支是master
  \item 自动为你的远程仓库,起一个名字,缺省是origin
\end{itemize}
\end{frame}

\begin{frame}[<+->][fragile]{回顾单独工作时的流程}
    \begin{itemize}
        \item 基于当前分支建一个临时分支,并切换到那里工作
        \item 修改代码 \(\ldots\) 提交
        \item 修改代码 \(\ldots\) 提交
        \item 切换回工作分支(master, dev, \(\ldots\))
        \item 把临时分支合并过来
        \item 删除临时分支(可保留再用)
        \item \(\ldots\)
    \end{itemize}
\end{frame}

\begin{frame}[<+->][fragile]{本地仓库推出去}
\onslide<+->
你的每一次提交,都存放在本地仓库中,别人如何知晓?
\onslide<+->
\begin{Verbatim}[frame=single,commandchars=\\\{\}]
git push
\end{Verbatim}
将本地仓库的内容“推送”到远程仓库

\onslide<+->
\medskip
自你上次和远程仓库同步以后,
\medskip

\onslide<+->
    远程仓库未曾受过别人的git push操作:
    
    你的git push将成功

\onslide<+->
\medskip
    反之,你的推送操作将会失败

    \begin{Verbatim}[frame=single,commandchars=\\\{\}]
! [rejected]    master -> master (non-fast-forward)
    \end{Verbatim}
\end{frame}

\begin{frame}[<+->][fragile]{合并远程仓库里的变更到本地}
\onslide<+->
用远程仓库的内容更新本地仓库
\begin{Verbatim}[frame=single,commandchars=\\\{\}]
git pull
\end{Verbatim}

\onslide<+->
以上操作暗中包含两个步骤
\begin{Verbatim}[frame=single,commandchars=\\\{\}]
git fetch  (from the remote ropository)
git merge  (the tracking branch to the topic branch)
\end{Verbatim}
\end{frame}

\begin{frame}[<+->][fragile]{合并只能在工作目录中进行}
\onslide<+->
git的逻辑是,合并永远不在远程仓库一端进行。
必须先在某人的工作目录中合并,提交到本地仓库,再推送到
远程。
\onslide<+->
如果本地合并一切顺利,合并之后,再次运行
\begin{Verbatim}[frame=single,commandchars=\\\{\}]
git push
\end{Verbatim}
\onslide<+->
如果合并出现冲突,解决冲突,提交改变到本地仓库,再推到远程。
\end{frame}

\begin{frame}[<+->][fragile]{分布式管理的体现}
\onslide<+->
git允许使用多个远程仓库。使用克隆命令时,自动为你指明一个名为origin的远程仓库。
以后可以再添加、删除、修改远程仓库。
\begin{Verbatim}[frame=single,commandchars=\\\{\}]
git remote 
git remote add github git://github.com/xxx/yyy.git
git remove
git rename
\end{Verbatim}

\onslide<+->
指明远程仓库名的操作
\begin{Verbatim}[frame=single,commandchars=\\\{\}]
git push github
git push github testbranch
git fetch github
\end{Verbatim}
\end{frame}

\begin{frame}[<+->][fragile]{平等}
\onslide<+->
理论上,没有那个仓库更高贵。

\onslide<+->
事实上,人们会将某些仓库理解成“正宗的”
\end{frame}
