\part{多人协同工作}
\begin{frame}[<+->][fragile]{远程仓库与本地仓库}
\begin{itemize}
  \item 本地仓库
  \item 远程仓库
  \item 克隆操作
\begin{Verbatim}[frame=single,commandchars=\\\{\}]
git clone ssh://example.com/git/our-project.git
\end{Verbatim}
\end{itemize}
\end{frame}

\begin{frame}[<+->][fragile]{仓库里有什么}
\begin{itemize}
  \item 一旦在本地得到仓库的副本,一切操作就都在本地运行
\begin{Verbatim}[frame=single,commandchars=\\\{\}]
git log
\end{Verbatim}
  \item 可以看到所有的提交历史,都被复制过来。
  \item 自动为你建立工作目录,缺省的分支是master
\end{itemize}
\end{frame}

\begin{frame}[<+->][fragile]{你的工作情景}
    \begin{itemize}
        \item 先建一个临时分支,并切换到那里
        \item 修改代码 \(\ldots\) 提交修改 \(\ldots\)
        \item 修改代码 \(\ldots\) 提交修改 \(\ldots\)
        \item 切换回主要工作分支(可以是master,但也未必)
        \item 把临时分支合并过来
        \item 删除原来的临时分支(或者把工作分支反过来合并到临时分支,在临时分支上继续工作)
        \item \(\ldots\)
    \end{itemize}
\end{frame}

\begin{frame}[<+->][fragile]{本地仓库推出去}
\onslide<+->
你所做的改动,统统提交到你本地的仓库,别人如何知晓?
\begin{Verbatim}[frame=single,commandchars=\\\{\}]
git push
\end{Verbatim}

\onslide<+->
如果自你上次和远程仓库同步以后,没有别人改动过远程的仓库,你的push不会遇到问题。
反之,推操作会失败。
\end{frame}

\begin{frame}[<+->][fragile]{合并远程仓库里的变更到本地}
\begin{Verbatim}[frame=single,commandchars=\\\{\}]
git fetch
git merge
\end{Verbatim}

\onslide<+->
或者,合成一步
\begin{Verbatim}[frame=single,commandchars=\\\{\}]
git pull
\end{Verbatim}
\end{frame}

\begin{frame}[<+->][fragile]{先听再说}
\onslide<+->
git的逻辑是,合并永远不在远程仓库一端进行。
必须到某个人的本地仓库,由他合并之后,再推倒
远程。
\onslide<+->
如果本地合并一切顺利,合并之后,再次运行
\begin{Verbatim}[frame=single,commandchars=\\\{\}]
git push
\end{Verbatim}
\onslide<+->
如果合并出现冲突,解决冲突,提交改变到本地仓库,再推到远程。
\end{frame}

\begin{frame}[<+->][fragile]{分布式管理的体现}
\onslide<+->
git允许使用多个远程仓库。使用克隆命令时,自动为你指明一个名为origin的远程仓库。
以后可以再添加、删除、修改远程仓库。
\begin{Verbatim}[frame=single,commandchars=\\\{\}]
git remote 
git remote add github git://github.com/xxx/yyy.git
git remove
git rename
\end{Verbatim}

\onslide<+->
指明远程仓库名的操作
\begin{Verbatim}[frame=single,commandchars=\\\{\}]
git push github
git push github testbranch
git fetch github
\end{Verbatim}
\end{frame}

\begin{frame}[<+->][fragile]{平等}
\onslide<+->
理论上,没有那个仓库更高贵。

\onslide<+->
事实上,人们会将某些仓库理解成“正宗的”
\end{frame}
