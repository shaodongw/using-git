\part{附带的话题}

\begin{frame}[<+->]{多人多项目}
    \begin{itemize}
        \item 情境

        Alan, Bob, Charlie, David 四个人

        X, Y, Z 三个项目
        \item 关系

\begin{center}
    \begin{tabular}{l|c c c}
        \hline
        Alan    &   X  &  Y  &  Z  \\ \hline
        Bob     &   X  &     &  Z  \\ \hline
        Charlie &      &  Y  &  Z  \\ \hline
        David   &      &     &  Z  \\ \hline
    \end{tabular}
\end{center}

        \item 解决思路

        每个项目设立一个Git用户
    \end{itemize}
\end{frame}


\begin{frame}[<+->][fragile]{钩子与自动化}
    \begin{itemize}
        \item 钩子的位置
        \item git clone 不复制钩子脚本
        \item 事件
        \item 脚本
        \item 范例
    \end{itemize}
\end{frame}

\begin{frame}[fragile]{参考材料}
    \begin{itemize}
        \item https://help.github.com/
        \item http://gitready.com/
        \item http://progit.org/
        \item http://gitimmersion.com/
    \end{itemize}
\end{frame}

\begin{frame}{获得本幻灯片}

{\Huge\tt https://github.com\\
/shaodongw/using-git\\
/blob/master\\
/using-git.pdf\\}
\end{frame}

\begin{frame}[<+->][fragile]{反馈}
{\huge\tt
shaodongwang@yahoo.com.cn
}
\end{frame}

